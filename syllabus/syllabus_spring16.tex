% Latex Document modified from Latex design of Brian R. Hall (available at www.brianrhall.net)

% Document settings
\documentclass[11pt]{article}
\usepackage[margin=1in]{geometry}
\usepackage[pdftex]{graphicx}
\usepackage{multirow}
\usepackage{setspace}
\usepackage[dvipsnames]{xcolor}
\pagestyle{plain}
\usepackage{color}
\setlength\parindent{0pt}
\makeatletter
\setlength{\@fptop}{0pt}
\makeatother

\begin{document}

% Course information
  \begin{center}

{\includegraphics[height=1.25in,width=1in]{wmchiffre1.jpg}} 

\LARGE Breaking Intuition: How Data has Changed Human Perception (Spring 2016, INTR 100)\\ \vspace{3mm}
\end{center}
\large \textbf{Schedule} \\
\normalsize \color{yellow}Gold \color{black} Section (CRN 24926): M, W - 12PM to 12:50PM, \textit{Morton Hall 244} \\
\color{green}Green \color{black} Section (CRN 24927): M, W - 1PM to 1:50PM, \textit{Morton Hall 244} \\
\color{gray}Silver \color{black} Section (CRN 24929): M, W - 2PM to 2:50PM, \textit{Morton Hall 244} \\
\textbf{All} \color{black} Sections : Friday, 12:00PM to 12:50PM, \textit{Morton Hall 220} \\
\vspace{2mm}

\begin{table}[ht]
\begin{tabular}{l l}
\large \textbf{Instructors} & \\
\large Dr. \textbf{Dan} Runfola & Dr. \textbf{Ty}ler Frazier \\
\large dsmillerrunfol@wm.edu  & tjfrazier@wm.edu \\
\large 424 Scotland Street & 424 Scotland Street \\
\large Office Hours: W 9-10A & Tue 2-4P, Thu 10-12P \\
\large 508.316.9109 & 757.386.1269 \\
\end{tabular}
\end{table}


\large \textbf{Teaching Assistants} \\
\large Name: Yaseen Lofti (yalotfi@email.wm.edu), Joel Monroe (jtmonroe@email.wm.edu), Vansh Bansal (vbansal@email.wm.edu)  \\
\large Office: Blow Hall Room 150 (SSRMC) \\
\large Directions: \\
http://ssrmc.wm.edu/wp-content/uploads/2014/08/Finding-the-SSRMC.jpg \\
\large Office Hours: 1P-5P \\


\textit{Please let me know if you have any documented disabilities that may impact your performance in this class.}

\textbf {\large \\ Course Description:} The human species has nearly always sought to predict the future – from reading the clouds and sacrificial rites, to statistics and big data parsing. At the core of this history lies human intuition, and the interplay between belief, story, and knowledge. In this class, students will learn the fundamental ways in which the human species has used data over time – from arcane interpretation to artificial intelligence – and closely inspect modern day assumptions about intuition and meaning. As a part of this course, students will learn how to parse and visualize large datasets to extract meaning, and use those findings to argue for or against solutions to real-world problems. \\
\\
\textbf {Prerequisite(s):} None.

\textbf {Credit Hours:} 4 \\

\textbf {\large Books you'll need to buy:}\\ \emph{Thinking, Fast and Slow}, any edition \\
Daniel Kahneman 
\vspace{2mm}

\emph{Automate This: How Algorithms Came to Rule Our World}, any edition \\
Christopher Steiner
\vspace{2mm}

\textbf {\large Media and Software available for free or on campus:}\\

Many TED talks\\
Visme - Online Software\\
R and RStudio - Free Software for Download\\
ArcGIS\\
Microsoft Excel\\

Throughout the course I may assign readings from academic or popular journals – these will be made available through either the Swem library or online via blackboard.

\vspace{8mm}

\textbf {\large Course Objectives:} 
\begin{enumerate} \itemsep-0.4em
  \item Challenge students traditional assumptions about the production of knowledge.
  \item Provide a more critical understanding of scientific literature.
  \item Develop your data communication, analysis, and visualization skills for future courses (and jobs).
  \item Develop critical thinking skills necessary to become more responsible for your own learning and better engage with broadly defined challenges.
  \item Provide an opportunity to fulfill the COLL 100 requirement (passing grade required).
\end{enumerate}
\vspace{8mm}

\textbf {\large Grade Distribution:} \\
\hspace*{40mm}
\begin{tabular}{ l l }
Lab Assignments & 80\% \\
Final Assignment & 20\% \\
\end{tabular} \\\\

\textbf {\large Letter Grade Distribution:} \\
\hspace*{40mm}
\begin{tabular}{ l l | l l }
\textgreater= 93.00 & A & 73.00 - 76.99 & C \\
90.00 - 92.99 & A-  & 70.00 - 72.99 & C- \\
87.00 - 89.99 & B+  & 67.00 - 69.99 & D+ \\
83.00 - 86.99 & B  & 63.00 - 66.99 & D \\
80.00 - 82.99 & B-  & 60.00 - 62.99 & D- \\
77.00 - 79.99 & C+  & \textless= 59.99 & F \\
\end{tabular} \\
.\\

\textbf {\large Time Commitment:} Excelling in college level course work typically requires on average three to four hours per credit per week.  Since this is a four credit course, in addition to the time we meet as a class each week, you should expect to spend nine to twelve hours on average reading, writing, or otherwise preparing for this class on a weekly basis.\\

\textbf {\large Attendance:} This class does not have an attendance policy.  However, it will be difficult to learn enough to pass the class without regular participation, as the majority of course content relevant to assignments will be covered in class.  Unannounced opportunities for extra credit will periodically be given in-class.\\

\textbf {\large Discussions:} Some lecture sessions will begin with a brief discussion of the assigned materials.  As such, each week students will be asked to come to class prepared with two to three bullet points for discussion - these can represent questions the material raised, commentary, or critiques.  These discussion points may be collected for extra credit.\\

\textbf {\large Classroom Behavior:} Please remain civil during discussions to promote the open exchange of ideas and foster a culture of open dialogue.  Please bear in mind that all students are entitled to their own opinion.  You are expected to listen attentively to each person speaking.  Please refrain from eating during class (and, if you must, make sure it isn't loud!).\\

\textbf {\large Teacher-student conferences:} Students performing at a C level or below are required to schedule a meeting with the instructor to discuss class performance.\\

\textbf {\large Late / Poor Performance Policy:} Assignments will not be accepted late, excepting in documented circumstances (i.e., an illness with a doctor's note).  \textit{Recognizing that it is entirely possible to under perform on key days, your lowest lab assignment grade will be dropped at the end of the semester. Your final assignment grade cannot be dropped.}\\

\textbf {\large Final Project:} The final project will be introduced the last few weeks, and will be due online by 11:59PM on the day of the scheduled final time (\textbf{Tuesday, May 10, 2016}). In this project, you will be assigned an open-ended question, and be asked to present a two to five page argument (including visualizations drawn from datasets) for why a certain course of action should - or should not - be taken. \\

\textbf {\large Important Dates:} The add and drop deadline this semester is Jan. 29, and withdrawal deadline is March 18. \\

\vspace{4mm}
\textbf {\LARGE Do not cheat!} \\
.\hrulefill . \\
\textbf{Academic dishonesty is taken very seriously.  Make sure to cite all of your work, and do not turn in work that is not yours!  Cases of academic dishonesty will be evaluated and acted upon in accordance with William and Mary policies, which can be found at http://www.wm.edu/offices/deanofstudents/services/
studentconduct/} \\
.\hrulefill . \\
\vspace{10mm}

% Course Outline
\newpage 
\textbf {\large Course Outline}:

The weekly content might change as it depends on the progress of the class.  You must keep up with the reading assignments.

\begin{table}[h!]
\small % The size of the table text can be changed depending on content. Remove if desired.
\begin{tabular}{ | c | c | }
\hline
\textbf{Week} & \textbf{Content} \\
\hline
Week 1 & \begin{minipage}{.85\textwidth}
\begin{itemize} \itemsep-0.4em
	\vspace{1mm}
	\item Wed, 1/20: An Introduction to R for Intelligent People
	\item Fri, 1/22 - Lecture 1.0 - Preparing to be wrong  \\ \textit{Watch: Ken Robinson - Do schools kill creativity?}
	\item \textbf{Lab 1 Due!}
	\vspace{1mm}
\end{itemize}
\end{minipage} \\
\hline
Week 2 & \begin{minipage}{.85\textwidth}
\begin{itemize} \itemsep-0.4em
	\vspace{1mm}
	\item Mon, 1/25; Wed, 1/27: A Bit more Advanced Introduction to R for Intelligent People
		\item Fri, 1/29: Lecture 2.0 - What is Intuition? \\ \textit{Reading: Thinking Fast and Slow, Chapter 1 - The characters of the story}\\
\textit{Watch: Apollo Robbins: The art of misdirection}
\item \textbf{Lab 2 due!}
	\vspace{1mm}
\end{itemize}
\end{minipage} \\
\hline
Week 3 & \begin{minipage}{.85\textwidth}
\begin{itemize} \itemsep-0.4em
	\vspace{1mm}
	\item Mon, 2/1; Wed, 2/3: Why William and Mary?
	\item Fri, 2/5: Lecture 3.0 - Storytelling and Knowledge \\ \textit{Thinking Fast and Slow, Chapter 6 - Norms, Surprises and Causes} 
	\vspace{1mm}
\end{itemize}
\end{minipage} \\
\hline
Week 4 & \begin{minipage}{.85\textwidth}
\begin{itemize} \itemsep-0.4em
	\vspace{1mm}
	\item Mon, 2/8; Wed, 2/10: Why William and Mary?
	\item Fri, 2/12: Lecture 4.0 - The History of Modern Intuition  
	\item Reading: TBD	
	\item \textit{Watch: Eric Sanderson: New York -- before the City}
	\item \textbf{Lab 3 due!}
	\vspace{1mm}
\end{itemize}
\end{minipage} \\
\hline
Week 5 & \begin{minipage}{.85\textwidth}
\begin{itemize} \itemsep-0.4em
	\vspace{1mm}
	\item Mon, 2/15; Wed, 2/17: GIS
	\item Fri, 2/19: Lecture 5.0: The Meaning of Big Data\\ 
	\textit{Automate This: How Algorithms Came to Rule Our World, Chapter 1: Wall Street, the First Domino}
	\item Reading: TBD
	\vspace{1mm}
\end{itemize}
\end{minipage} \\
\hline
Week 6 & \begin{minipage}{.85\textwidth}
\begin{itemize} \itemsep-0.4em
	\vspace{1mm}
	\item Mon, 2/22; Wed, 2/24: GIS
	\item Fri, 2/26: Lecture 6.0: Applications of Knowledge \\ \textit{The Hidden Influence of Social Networks - Nicholas Christakis}\\
	\textit{Reading: Connected, Chapter 1 - In the Thick of It}
	\item \textbf{Lab 4 due!}
	\vspace{1mm}
\end{itemize}
\end{minipage} \\
\hline
Week 7 & \begin{minipage}{.85\textwidth}
\begin{itemize} \itemsep-0.4em
	\vspace{1mm}
	\item Mon, 2/29; Wed, 3/2: Business Analytics
	\item Fri, 3/4: Guest Lecture - Dr. Rob Rose, Center for Geospatial Analysis
	\vspace{1mm}
\end{itemize}
\end{minipage} \\
\hline
Week 8 & \begin{minipage}{.85\textwidth}
\begin{itemize} \itemsep-0.4em
	\vspace{1mm}
	\item Mon, 3/7: No Class (Spring Break)
	\item Wed, 3/9: No Class (Spring Break)
	\item Fri, 3/11: No Class (Spring Break)
	\vspace{1mm}
\end{itemize}
\end{minipage} \\
\hline
Week 9 & \begin{minipage}{.85\textwidth}
\begin{itemize} \itemsep-0.4em
	\vspace{1mm}
	\item Mon, 3/14; Wed 3/16: Business Analytics
	\item Fri, 3/18: Guest Lecture - Salil Singhal, Bank Strategy, Capital One
	\item \textbf{Lab 5 due!}
	\vspace{1mm}
\end{itemize}
\end{minipage} \\
\hline
Week 10 & \begin{minipage}{.85\textwidth}
\begin{itemize} \itemsep-0.4em
	\vspace{1mm}
		\item Mon, 3/21; Wed 3/23: Social Networks
	\item Fri, 3/25: Lecture 7: Why we're generally wrong  \\ \textit{Thinking, Fast and Slow - Chapter 18: Taming Intuitive Predictions}
		\textit{Reading: Connected Chapter 2 - When you Smile, The World Smiles With You}
	\vspace{1mm}
\end{itemize}
\end{minipage} \\
\hline
Week 11 & \begin{minipage}{.85\textwidth}
\begin{itemize} \itemsep-0.4em
	\item Mon, 3/28; Wed 3/30: Social Networks
	\item Fri, 4/1: Guest Lecture: Jaime Settle - The Power of Social Networks\\
	\item \textbf{Lab 6 due!}
	\vspace{1mm}
\end{itemize}
\end{minipage} \\
\hline
\end{tabular} 
\end{table}
\newpage
\begin{table}[h!]
\small 
\begin{tabular}{ | c | c | }

\hline
Week 12 & \begin{minipage}{.85\textwidth}
\begin{itemize} \itemsep-0.4em
	\vspace{1mm}
	\item Mon, 4/4; Wed 4/6: Development Geography
	\item Fri, 4/8: "Guest" Lecture: Tyler Frazier, AidData
	\vspace{1mm}
\end{itemize}
\end{minipage} \\
\hline
Week 13 & \begin{minipage}{.85\textwidth}
\begin{itemize} \itemsep-0.4em
	\vspace{1mm}
	\item Mon, 4/11; Wed 4/13: Development Geography
	\item Fri, 4/15: Guest Lecture: Dan Parker\\
	\item \textbf{Lab 7 due!}
	\vspace{1mm}
\end{itemize}
\end{minipage} \\

\hline
Week 14 & \begin{minipage}{.85\textwidth}
\begin{itemize} \itemsep-0.4em
	\vspace{1mm}
	\item Mon, 4/18, 4/20: Final Project - Thinking Like Machines
	\item Fri, 4/22: Lecture 8: Perception and Data Visualization
	\textit{Read: Automate This - Chapter 7 - Categorizing Humankind}
	\vspace{1mm}
\end{itemize}
\end{minipage} \\
\hline
Week 15 & \begin{minipage}{.85\textwidth}
\begin{itemize} \itemsep-0.4em
	\vspace{1mm}
	\item Mon, 4/25, Wed 4/27: Final Project - Thinking Like Machines 
	\item Fri, 4/29: Lecture 9: Breaking Intuition  \\ \textit{Thinking, Fast and Slow - Chapter 36: Life as a Story;\\ Automate This - Chapter 10 - The Future} 
	\item \textbf{Final Lab due by 11:59PM on Tuesday, May 10, 2016}
	\vspace{1mm}
\end{itemize}
\end{minipage} \\

\hline

\end{tabular} 
\end{table}

\end{document}



